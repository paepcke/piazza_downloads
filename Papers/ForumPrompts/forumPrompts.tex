% THIS IS SIGPROC-SP.TEX - VERSION 3.1
% WORKS WITH V3.2SP OF ACM_PROC_ARTICLE-SP.CLS
% APRIL 2009
%
% It is an example file showing how to use the 'acm_proc_article-sp.cls' V3.2SP
% LaTeX2e document class file for Conference Proceedings submissions.
% ----------------------------------------------------------------------------------------------------------------
% This .tex file (and associated .cls V3.2SP) *DOES NOT* produce:
%       1) The Permission Statement
%       2) The Conference (location) Info information
%       3) The Copyright Line with ACM data
%       4) Page numbering
% ---------------------------------------------------------------------------------------------------------------
% It is an example which *does* use the .bib file (from which the .bbl file
% is produced).
% REMEMBER HOWEVER: After having produced the .bbl file,
% and prior to final submission,
% you need to 'insert'  your .bbl file into your source .tex file so as to provide
% ONE 'self-contained' source file.
%
% Questions regarding SIGS should be sent to
% Adrienne Griscti ---> griscti@acm.org
%
% Questions/suggestions regarding the guidelines, .tex and .cls files, etc. to
% Gerald Murray ---> murray@hq.acm.org
%
% For tracking purposes - this is V3.1SP - APRIL 2009

\documentclass{edm_template}
\usepackage{multirow}
\usepackage{csquotes}
\usepackage{listings}
\usepackage{fixltx2e}
\usepackage{floatrow}
\usepackage{url}
\usepackage{mathtools}
\usepackage{placeins}
\usepackage[font=small,skip=0pt]{caption}

\makeatletter
\def\@copyrightspace{\relax}
\makeatother
\setlength{\floatsep}{5pt}
\setlength{\textfloatsep}{4pt}
\setlength{\intextsep}{5pt}

\newcommand{\squishlist}{
 \begin{list}{$\bullet$}
 {
  \setlength{\itemsep}{0pt}
  \setlength{\parsep}{3pt}
  \setlength{\topsep}{3pt}
  \setlength{\partopsep}{0pt}
  \setlength{\leftmargin}{1.5em}
  \setlength{\labelwidth}{1em}
  \setlength{\labelsep}{0.5em} } }

\newcommand{\squishlisttwo}{
 \begin{list}{$\bullet$}
 {
  \setlength{\itemsep}{0pt}
  \setlength{\parsep}{0pt}
  \setlength{\topsep}{0pt}
  \setlength{\partopsep}{0pt}
  \setlength{\leftmargin}{2em}
  \setlength{\labelwidth}{1.5em}
  \setlength{\labelsep}{0.5em} } }

\newcommand{\squishend}{
  \end{list}  }

\begin{document}

\title{Supporting the Encouragement of Forum Participation}
#\subtitle{[Extended Abstract]
#\titlenote{A full version of this paper is available at
#\texttt{http://ilpubs.stanford.edu:8090/1140/1/indexer.pdf}}}
%
% You need the command \numberofauthors to handle the 'placement
% and alignment' of the authors beneath the title.
%
% For aesthetic reasons, we recommend 'three authors at a time'
% i.e. three 'name/affiliation blocks' be placed beneath the title.
%
% NOTE: You are NOT restricted in how many 'rows' of
% "name/affiliations" may appear. We just ask that you restrict
% the number of 'columns' to three.
%
% Because of the available 'opening page real-estate'
% we ask you to refrain from putting more than six authors
% (two rows with three columns) beneath the article title.
% More than six makes the first-page appear very cluttered indeed.
%
% Use the \alignauthor commands to handle the names
% and affiliations for an 'aesthetic maximum' of six authors.
% Add names, affiliations, addresses for
% the seventh etc. author(s) as the argument for the
% \additionalauthors command.
% These 'additional authors' will be output/set for you
% without further effort on your part as the last section in
% the body of your article BEFORE References or any Appendices.

\numberofauthors{2} %  in this sample file, there are a *total*
% of EIGHT authors. SIX appear on the 'first-page' (for formatting
% reasons) and the remaining two appear in the \additionalauthors section.
%
\author{
% You can go ahead and credit any number of authors here,
% e.g. one 'row of three' or two rows (consisting of one row of three
% and a second row of one, two or three).
%
% The command \alignauthor (no curly braces needed) should
% precede each author name, affiliation/snail-mail address and
% e-mail address. Additionally, tag each line of
% affiliation/address with \affaddr, and tag the
% e-mail address with \email.
%
% 1st. author
\alignauthor Aashna Garg \\
       \affaddr{Stanford University} \\
       \email{aashna94@stanford.edu}
\alignauthor Andreas Paepcke \\
       \affaddr{Stanford University} \\
       \email{paepcke@cs.stanford.edu }
% \and   use '\and' if you need 'another row' of author names
} % \author
\date{21 February 2017}
% Just remember to make sure that the TOTAL number of authors
% is the number that will appear on the first page PLUS the
% number that will appear in the \additionalauthors section.

\maketitle
\begin{abstract}
**** Abstract goes here
\end{abstract}

%% A category with the (minimum) three required fields
%\category{H.4}{Information Systems Applications}{Miscellaneous}
%%A category including the fourth, optional field follows...
%\category{D.2.8}{Software Engineering}{Metrics}[complexity measures, performance measures]
%
%\terms{Theory}

% TODO:
%   o Rose mentions centrality; important for us?

\section{Introduction}

- Why forum important
    o For humanities
    o For helping each other
- Not all participate
    o why?
    o Observing actions every week to predict dropout.
      Maybe had forum actions?:
      Balakrishnan Girish. Predicting Student Retention in Massive
      Open Online Courses using Hidden MarkovModels. EECS Department,
      University of California, Berkeley, May, 2013.   

- Residential is different from MOOCs:
    o Incentives: more at stake after drop-out
    o Problem of forum cliques falling apart from
      course attrition goes away.
    o No late-comers that stay at the periphery
        (Rose:Turn-off)
    o Possible prior acquaintance
    o Smaller
- We have evidence that encouragement tricky
    o Grade: reduces intrinsic motivation.
    o Appeal to personal growth: negative
    o Appeal to community ?
    o These interventions were at the start, when
        urgency not there yet.
- Now: many courses available: reruns and varied
    o We can see dynamics of forum participation
      over time through a course. For top and
      average students.
    o When do prolific students tend to get engaged?
    o At any point in time, can we see a
      trajectory of prolific and median students?

- Analyzed the graph, snapshots over time. Are there
  inflection points?

    o Cite
      [14] Borgatti, Stephen P. Centrality and network flow. Social
      networks 27.1 (2005): 55-71. Rose uses their interpretation of
      what the social graphs mean 
    o Identify places where encouragement could be given.
      Inflection points?

    o We don't discuss the *form* of encouragement, though
      the messaging is important:

      * These say that emotional support is important:

        [13] Wang, Yi-Chia, Robert Kraut, and John M. Levine. To stay
        or leave?: the relationship of emotionaland informational support
        to commitment in online health support groups. Proceedings of the
        ACM 2012conference on Computer Supported Cooperative Work. ACM,
        2012.       
      * Could appeal to personal gain for forum participation,
      * Could appeal to community gain for forum participation,
      * Could just state facts.

\section{From Posts to Connection Graph}




% \input{relatedWork.tex}
% \input{goldIndex.tex}
%\input{experiments.tex}
%\input{tfidf.tex}
%\input{includingPOS.tex}
%\input{includingWikipedia.tex}
%\input{resultsSummary.tex}
% \input{discussion.tex}
%\input{conclusion.tex}

%\bibliographystyle{plain}
\bibliographystyle{abbrv}
\bibliography{forumPrompts}

\end{document}
